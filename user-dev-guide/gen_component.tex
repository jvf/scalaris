\section{\texorpdfstring{The \erlmodule{gen\_component}}
         {The gen\_component}}
\erlmoduleindex{gen\_component}
\svnrev{r2675}

The generic component model implemented by
\erlmodule[noindex]{gen\_component} allows to add some common functionality
to all the components that build up the \scalaris{} system. It supports:

\begin{description}
\setlength{\parskip}{0pt}
\setlength{\itemsep}{0pt}
\item[event-handlers:] message handling with a similar syntax as used in \cite{rachid-book}.
\item[FIFO order of messages:] components cannot be inadvertently locked as
  we do not use selective receive statements in the code.
\item[sleep and halt:] for testing components can sleep or be halted.
\item[debugging, breakpoints, stepwise execution:] to debug components
  execution can be steered via breakpoints, step-wise execution and
  continuation based on arriving events and user defined component state
  conditions.
\item[basic profiling,]
\item[state dependent message handlers:] depending on its state, different
  message handlers can be used and switched during runtime. Thereby a kind
  of state-machine based message handling is supported.
\item[prepared for \erlmodule{pid\_groups}:] allows to send events to
  named processes inside the same group as the actual component itself
  (\code{send_to_group_member}) when just holding a reference to any group
  member, and
\item[unit-testing of event-handlers:] as message handling is separated from
  the main loop of the component, the handling of individual messages and
  thereby performed state manipulation can easily tested in unit-tests by
  directly calling message handlers.
\end{description}

In \scalaris{} all Erlang processes should be implemented as
\erlmodule{gen\_component}. The only exception are functions interfacing to
the client, where a transition from asynchronous to synchronous request
handling is necessary and that are executed in the context of a client's
process or a process that behaves as a proxy for a client
(\erlmodule{cs\_api}).

\subsection{\texorpdfstring{A basic \erlmodule{gen\_component} including a message handler}
             {A basic gen\_component including a message handler}}

To implement a \erlmodule{gen\_component}, the component has to provide the
\erlmodule{gen\_component} behaviour:

\codesnippet{gen_component.erl}{gen_component:behaviour}{../src/gen_component.erl}

This is illustrated by the following example:

\codesnippet{msg_delay.erl}{gen_component:sample}{../src/msg_delay.erl}

\erlfun{your\_gen\_component}{init}{/1} is called during start-up of a
\erlmodule{gen\_component} and should return the initial state to be used
for this \erlmodule{gen\_component}. Later, the current state of the
component can be retrieved using \erlfun{gen\_component}{get\_state}{/1}.

To react on messages / events, a message handler is used. The default
message handler is given to \erlfun{gen\_component}{start\_link}{/3} or
\erlfun{gen\_component}{start\_link}{/4} as well as
\erlfun{gen\_component}{start}{/3}, \erlfun{gen\_component}{start}{/4} or
\erlfun{gen\_component}{start}{/5}. It can be
changed by calling \erlfun{gen\_component}{change\_handler}{/2} (see
Section~\ref{sec:gen_component:change_handler}). When an event / message for
the component arrives, this handler is called with the event itself and the
current state of the component. In the handler, the state of the component
may be adjusted depending upon the event. The handler itself may trigger new
events / messages for itself or other components and has finally to return
the updated state of the component or the atoms \texttt{unknown\_event} or
\texttt{kill}. It must neither call \code{receive} nor
\erlfun{timer}{sleep}{/1} nor \erlfun{erlang}{exit}{/1}.

\subsection{\texorpdfstring{How to start a \erlmodule{gen\_component}?}
             {How to start a gen\_component?}}

A \erlmodule{gen\_component} can be started using one of:

\erlfun{gen\_component}{start}{Module, Args, GenCOptions = []}\\
\erlfun{gen\_component}{start\_link}{Module, Args, GenCOptions = []}
\begin{erlfunparams}
\erlparam{Module}{the name of the module your component is
   implemented in}
\erlparam{Args}{List of parameters passed to \code{Module:init/1}
   for initialization}
\erlparam{GenCOptions}{optional parameter.
   List of options for \erlmodule{gen\_component}
\begin{description}
\setlength{\parskip}{0pt}
\setlength{\itemsep}{0pt}
\item[\texttt{\{pid\_groups\_join\_as, ProcessGroup, ProcessName\}}:] registers the new
  process with the given process group (also called instanceid) and name
  using \erlmodule{pid\_groups}.
\item[\texttt{\{erlang\_register, ProcessName\}}:] registers the process as
  a named Erlang process.
\item[\texttt{\{wait\_for\_init\}}:] wait for \code{Module:init/1} to return
  before returning to the caller.
\end{description}
}
\end{erlfunparams}

These functions are compatible to the Erlang/OTP supervisors.  They spawn a
new process for the component which itself calls \code{Module:init/1} with
the given \code{Args} to initialize the component. \code{Module:init/1}
should return the initial state for your component. For each message sent to
this component, the default message handler
\erlfun[noindex]{Module}{on}{Message, State} will be called, which should
react on the message and return the updated state of your component.

\erlfun{gen\_component}{start}{} and \erlfun{gen\_component}{start\_link}{}
return the pid of the spawned process as \code{\{ok, Pid\}}.


\subsection{\texorpdfstring{When does a \erlmodule{gen\_component} terminate?}
             {When does a gen\_component terminate?}}

A \erlmodule{gen\_component} can be stopped using:

\erlfun{gen\_component}{kill}{Pid} or by returning \code{kill} from the
current message handler.

\subsection{\texorpdfstring{How to determine whether a process is a \erlmodule{gen\_component}?}
             {How to determine whether a process is a gen\_component?}}

A \erlmodule{gen\_component} can be detected by:

\erlfun{gen\_component}{is\_gen\_component}{Pid}, which returns a boolean.

\subsection{What happens when unexpected events / messages arrive?}

Your message handler (default is \erlfun{your\_gen\_component}{on}{/2})
should return \code{unknown_event} in the final clause
(\erlfun{your\_gen\_component}{on}{\_,\_}).  \erlmodule{gen\_component} then
will nicely report on the unhandled message, the component's name, its state
and currently active message handler, as shown in the following example:

\begin{lstlisting}[language=bash]
# bin/boot.sh
[...]
(boot@localhost)10> pid_groups ! {no_message}.
{no_message}
[error] unknown message: {no_message} in Module: pid_groups and
handler on in State null
(boot@localhost)11>
\end{lstlisting}

The \erlmodule{pid\_groups} (see
Section~\ref{sec:pid_groups}) is a \erlmodule{gen\_component} which
registers itself as named Erlang process with the \erlmodule{gen\_component}
option \code{erlang_register} and therefore can be addressed by its name in
the Erlang shell. We send it a \code{\{no_message\}} and
\erlmodule{gen\_component} reports on the unhandled message. The
\erlmodule{pid\_groups} module itself continues to run and waits for
further messages.

\subsection{What if my message handler generates an exception or
 crashes the process?}

\erlmodule{gen\_component} catches exceptions generated by message handlers
and reports them with a stack trace, the message, that generated the
exception, and the current state of the component.

If a message handler terminates the process via \erlfun{erlang}{exit}{/1},
this is out of the responsibility scope of \erlmodule{gen\_component}. As
usual in Erlang, all linked processes will be informed. If for example
\erlfun{gen\_component}{start\_link}{/2} or \texttt{/3} was used for
starting the \erlmodule{gen\_component}, the spawning process will be
informed, which may be an Erlang supervisor process taking further actions.

\subsection{Changing message handlers and implementing state dependent
 message responsiveness as a state-machine}
\label{sec:gen_component:change_handler}
\erlfunindex{gen\_component}{change\_handler}

Sometimes it is beneficial to handle messages depending on the state of a
component. One possibility to express this is implementing different clauses
depending on the state variable, another is introducing case clauses inside
message handlers to distinguish between current states. Both approaches may
become tedious, error prone, and may result in confusing source code.

Sometimes the use of several different message handlers for different states
of the component leads to clearer arranged code, especially if the set of
handled messages changes from state to state. For example, if we have a
component with an initialization phase and a production phase afterwards, we
can handle in the first message handler messages relevant during the
initialization phase and simply queue all other requests for later
processing using a common default clause.

When initialization is done, we handle the queued user requests and switch
to the message handler for the production phase. The message handler for the
initialization phase does not need to know about messages occurring during
production phase and the message handler for the production phase does not
need to care about messages used during initialization. Both handlers can be
made independent and may be extended later on without any adjustments to the
other.

One can also use this scheme to implement complex state-machines by changing
the message handler from state to state.

To switch the message handler
\erlfun{gen\_component}{change\_handler}{State, new\_handler} is called as
the last operation after a message in the active message handler was
handled, so that the return value of
\erlfun{gen\_component}{change\_handler}{/2} is propagated to
\erlmodule{gen\_component}.  The new handler is given as an atom, which is
the name of the 2-ary function in your component module to be called.

\subsubsection{Starting with non-default message handler.}
It is also possible to change the message handler right from the start
in your \erlfun{your\_gen\_component}{init}{/1} to avoid the default message
handler \erlfun{your\_gen\_component}{on}{/2}. Just create your initial
state as usual and call \erlfun{gen\_component}{change\_handler}{State,
  my\_handler} as the final call in your
\erlfun{your\_gen\_component}{init}{/1}. We prepared
\erlfun{gen\_component}{change\_handler}{/2} to return \code{State} itself,
so this will work properly.

\subsection{Handling several messages atomically}

The message handler is called for each message separately. Such a single
call is atomic, i.e. the component does not perform any other action until
the called message handler finishes. Sometimes, it is necessary to execute
two or more calls to the message handler atomically (without other
interleaving messages). For example if a message \code{A} contains another
message \code{B} as payload, it may be necessary to handle \code{A} and
\code{B} directly one after the other without interference of other
messages. So, after handling \code{A} you want to call your message handler
with \code{B}.

In most cases, you could just do so by calculating the new state as result
of handling message \code{A} first and then calling the message handler with
message \code{B} and the new state by yourself.

It is safer to use \erlfun{gen\_component}{post\_op}{2} in such cases: When
$B$ contains a special message, which is usually handled by the
\erlmodule{gen\_component} module itself (like
\code{send\_to\_group\_member}, \code{kill}, \code{sleep}), the direct call
to the message handler would not achieve the expected result. By calling
\erlfun{gen\_component}{post\_op}{B, NewState} to return the new state after
handling message \code{A}, message \code{B} will be handled directly after
the current message \code{A}.

\subsection{\texorpdfstring{Halting and pausing a \erlmodule{gen\_component}}
{Halting and pausing a gen\_component}}

Using \erlfun{gen\_component}{kill}{Pid} and
\erlfun{gen\_component}{sleep}{Pid, Time} components can be terminated or
paused.

\subsection{\texorpdfstring{Integration with \erlmodule{pid\_groups}:
  Redirecting messages  to other \erlmodule{gen\_component}s}
  {Integration with pid\_groups:
  Redirecting messages  to other gen\_components}}

Each \erlmodule{gen\_component} by itself is prepared to support
\erlfun{comm}{send\_to\_group\_member}{/3} which forwards messages inside a
group of processes registered via \erlmodule{pid\_groups} (see
Section~\ref{sec:pid_groups}) by their name. So, if you hold a Pid
of one member of a process group, you can send messages to other members of
this group, if you know their registered Erlang name. You do not necessarily
have to know their individual Pid.

\emph{In consequence, no \erlmodule{gen\_component} can individually handle
  messages of the form \code{\{send_to_group_member,} \code{_, _\}} as such
  messages are consumed by \erlmodule{gen\_component} itself.}

\subsection{\texorpdfstring{Replying to \code{ping} messages}
  {Replying to ping messages}}

Each \erlmodule{gen\_component} replies automatically to \code{\{ping,
  Pid\}} requests with a \code{\{pong\}} send to the given \code{Pid}.  Such
messages are generated, for example, by \erlmodule{vivaldi\_latency} which is used
by our \erlmodule{vivaldi} module.

\emph{In consequence, no \erlmodule{gen\_component} can individually handle
messages of the form: \code{\{ping, _\}} as such messages are consumed by
\erlmodule{gen\_component} itself.}


\subsection{\texorpdfstring{The debugging interface of \erlmodule{gen\_component}:
  Breakpoints and step-wise execution}
  {The debugging interface of gen\_component: Breakpoints and step-wise execution}}

We equipped \erlmodule{gen\_component} with a debugging interface, which
especially is beneficial, when testing the interplay between several
\erlmodule{gen\_component}s. It supports breakpoints (bp) which can pause the
\erlmodule{gen\_component} depending on the arriving messages or depending
on user defined conditions. If a breakpoint is reached, the execution can be
continued step-wise (message by message) or until the next breakpoint is
reached.

We use it in our unit tests to steer protocol interleavings and to perform
tests using random protocol interleavings between several processes
(see~\erlmodule{paxos\_SUITE}). It allows also to reproduce given protocol
interleavings for better testing.

\subsubsection{Managing breakpoints.}

Breakpoints are managed by the following functions:

\begin{description}
\item[\erlfun{gen\_component}{bp\_set}{Pid, MsgTag, BPName}:] For the
  component running under \code{Pid} a breakpoint \code{BPName} is set. It
  is reached, when a message with a message tag \code{MsgTag} is next to be
  handled by the component (See \erlfun{comm}{get\_msg\_tag}{/1} and
  Section~\ref{sec:comm} for more information on message tags). The
  \code{BPName} is used as a reference for this breakpoint, for example to
  delete it later.
\item[\erlfun{gen\_component}{bp\_set\_cond}{Pid, Cond, BPName}:]
  The same as \erlfun{gen\_component}{bp\_set}{/3} but a user defined
  condition implemented in \code{\{Module, Function, Params = 2\} = Cond} is
  checked by calling \code{Module:Function(Message, State)} to decide
  whether a breakpoint is reached or not. \code{Message} is the next message
  to be handled by the component and \code{State} is the current state of
  the component. \code{Module:Function/2} should return a \code{boolean}.
\item[\erlfun{gen\_component}{bp\_del}{Pid, BPName}:] The breakpoint
  \code{BPName} is deleted. If the component is in this breakpoint, it will
  not be released by this call. This has to be done separately by
  \erlfun{gen\_component}{bp\_cont}{/1}. But the deleted breakpoint will no
  longer be considered for newly entering a breakpoint.
\item[\erlfun{gen\_component}{bp\_barrier}{Pid}:]
  Delay all further handling of breakpoint requests until a breakpoint is
  actually entered.

  \emph{Note, that the following call sequence may not catch the breakpoint at
  all, as during the sleep the component not necessarily consumes a
  \code{ping} message and the set breakpoint `\code{sample_bp}' may already
  be deleted before a ping message arrives.}

  \begin{lstlisting}
  gen_component:bp_set(Pid, ping, sample_bp),
  timer:sleep(10),
  gen_component:bp_del(Pid, sample_bp),
  gen_component:bp_cont(Pid).
  \end{lstlisting}

  \emph{To overcome this, \erlfun{gen\_component}{bp\_barrier}{/1} can be used:}

  \begin{lstlisting}
  gen_component:bp_set(Pid, ping, sample_bp),
  gen_component:bp_barrier(Pid),
  %% After the bp_barrier request, following breakpoint requests
  %% will not be handled before a breakpoint is actually entered.
  %% The gen_component itself is still active and handles messages as usual
  %% until it enters a breakpoint.
  gen_component:bp_del(Pid, sample_bp),
  % Delete the breakpoint after it was entered once (ensured by bp_barrier).
  % Release the gen_component from the breakpoint and continue.
  gen_component:bp_cont(Pid).
  \end{lstlisting}
\end{description}

None of the calls in the sample listing above is blocking. It just schedules
all the operations, including the \code{bp_barrier}, for the
\erlmodule{gen\_component} and immediately finishes. The actual events of
entering and continuing the breakpoint in the \erlmodule{gen\_component}
happens independently later on, when the next \code{ping} message arrives.

\subsubsection{Managing execution.}

The execution of a \erlmodule{gen\_component} can be managed by the
following functions:

\begin{description}
\item[\erlfun{gen\_component}{bp\_step}{Pid}:] This is the only blocking
  breakpoint function. It waits until the \erlmodule{gen\_component} is in a
  breakpoint and has handled a single message.  It returns the module, the
  active message handler, and the handled message as a tuple \code{\{Module,
    On, Message\}}.  This function does not actually finish the breakpoint,
  but just lets a single message pass through. For further messages, no
  breakpoint condition has to be valid, the original breakpoint is still
  active. To leave a breakpoint, use \erlfun{gen\_component}{bp\_cont}{/1}.

\item[\erlfun{gen\_component}{bp\_cont}{Pid}:]
  Leaves a breakpoint. \erlmodule{gen\_component} runs as usual until the
  next breakpoint is reached.

  If no further breakpoints should be entered after continuation, you should
  delete the registered breakpoint using \erlfun{gen\_component}{bp\_del}{/2}
  before continuing the execution with
  \erlfun{gen\_component}{bp\_cont}{/1}. To ensure, that the breakpoint is
  entered at least once, \erlfun{gen\_component}{bp\_barrier}{/1} should be
  used before deleting the breakpoint (see the example above). Otherwise it
  could happen, that the delete request arrives at your
  \erlmodule{gen\_component} before it was actually triggered. The following
  continuation request would then unintentional apply to an unrelated
  breakpoint that may be entered later on.

\item[\erlfun{gen\_component}{runnable}{Pid}:] Returns whether a
  \erlmodule{gen\_component} has messages to handle and is runnable. If you
  know, that a \erlmodule{gen\_component} is in a breakpoint, you can use
  this to check, whether a \erlfun{gen\_component}{bp\_step}{/1} or
  \erlfun{gen\_component}{bp\_cont}{/1} is applicable to the component.

\end{description}

\subsubsection{Tracing handled messages -- getting a message  interleaving protocol.}

We use the debugging interface of \erlmodule{gen\_component} to test
protocols with random interleaving. First we start all the components
involved, set breakpoints on the initialization messages for a new Paxos
consensus and then start a single Paxos instance on all of them. The outcome
of the Paxos consensus is a \code{learner_decide} message. So, in
\erlfun{paxos\_SUITE}{step\_until\_decide}{/3} we look for runnable processes
and select randomly one of them to perform a single step until the protocol
finishes with a decision.

\codesnippet{paxos_SUITE.erl}{paxos_SUITE:random_interleaving_test}{../test/paxos_SUITE.erl}

To get a message interleaving protocol, we either can output the results
of each \erlfun{gen\_component}{bp\_step}{/1} call together with the
Pid we selected for stepping, or alter the definition of the macro
\code{TRACE_BP_STEPS} in \erlmodule{gen\_component}, when we execute all
\erlmodule{gen\_component}s locally in the same Erlang virtual machine.

\codesnippet{gen_component.erl}{gen_component:trace_bp_steps}{../src/gen_component.erl}


%% \subsection{Profiling \erlmodule{gen\_component}s.}
%% 
%% Using the profiling feature, see also Section~\ref{sec:digging} into the
%% details'
%% 
%% The profiling currently measures global time, which is actually not a
%% good metric for profiling, as it depends on the scheduling and not all
%% time recorded is actually used by the handler. Additionally
%% \erlfun{erlang}{now}{} seems to be slow, as it involves a system call to
%% get a monotonic clock information (which internally triggers a lock
%% across all Erlang execution threads at least in the Erlang runtime
%% environment up to R1304).
%% 
%% \todo{The metric should be changed to something more
%%   appropriate. Probably the number of reductions this process has done
%%   (via \erlfun{erlang}{process\_info}{}).}
%% 
%% \todo{
%%   Maybe we should add some functions as interface to read the
%%   profiling results instead of expecting the user to know how to read
%%   it directly from the \erlmodule{ets} table.}

\subsection{\texorpdfstring{Future use and planned extensions for \erlmodule{gen\_component}}
             {Future use and planned extensions for gen\_component}}

\erlmodule{gen\_component} could be further extended. For example it could
support hot-code upgrade or could be used to implement algorithms that
have to be run across several components of \scalaris{} like snapshot
algorithms or similar extensions.

%% \subsection{The \erlmodule{gen\_component} API for reference}



%%% Local Variables: 
%%% mode: latex
%%% End: 

